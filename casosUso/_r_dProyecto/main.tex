\documentclass[fleqn]{article}

\usepackage{graphicx}
\usepackage[utf8]{inputenc}
\usepackage{amsmath}
\usepackage{amssymb}
\usepackage{mathtools}
\usepackage[margin=3cm]{geometry}
\usepackage[ampersand]{easylist}

\title{\huge Cati Software\\ \Huge -R-D Proyecto\\ \Large Caso de Uso \vspace{30pt}}
\author{Ignacio Ampuero\\ 201473032-2 \and Anghelo Carvajal\\ 201473062-4 \and Vicente Lizana\\ 201310004-K}
\date{\today}

\renewcommand{\rmdefault}{lmss}

\begin{document}

\maketitle
\vspace{30pt}

\section{Actores}

	\subsection{Primario}
	
	\begin{itemize}
		\item Administrador
	\end{itemize}
	
	\subsection{Secundarios}
	
	\subsection{Off Stage}
	
	\begin{itemize}
		\item Cliente
	\end{itemize}

\section{Precondiciones}

\section{Postcondiciones}

\begin{itemize}
	\item Al borrar un proyecto, cualquier información que no haya sido guardada manualmente se perderá de manera irreversible.
\end{itemize}

\section{Curso Normal}

\begin{easylist}
	& Administrador selecciona ``Ver proyectos".
	& Administrador selecciona un proyecto en particular.
	& Sistema muestra detalles y opciones del proyecto.
	& Al finalizar, Administrador vuelve al menú.
\end{easylist}

\section{Curso alternativo}

\begin{easylist}
	\ListProperties(Start1=3)
	&& Administrador desea eliminar proyecto.
	&&& Administrador presiona ``Eliminar proyecto".
	&&& Sistema solicita clave de Administrador.
	&&& Administrador ingresa clave.
\end{easylist}

\end{document}
